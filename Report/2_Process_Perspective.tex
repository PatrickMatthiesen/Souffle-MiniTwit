\section{Process' Perspective}

A description and illustration of:

\subsection{Collaborative Development \& Team Organization}

When we initially talked about teaming up we made sure to align our expectations. We agreed that each member expected to spend 8-12 hours a week on the course. Thus our way of working became something like this:
\begin{itemize}
    \item Tuesday, 10-14: Project Work
    \item Tuesday, 14-16: Lecture
    \item Tuesday, 16-18: Exercises/Project Work
    \item Friday, 10-15: Project Work (When possible, not each Friday)
\end{itemize}

\noindent The above points were all physically present sessions for all members. Each member had the freedom to do additional work at other times during the week. All communication was through a Discord channel.

\noindent When working, we would at times do versions of pair programming. Teamwork was however difficult, since all concepts were new to everybody, and a lot of the understanding and progress came from trial and error. All sessions were an open forum where questions were welcome, and help was always possible to get.

\noindent We utilized GitHub Issues to share the tasks between us, making sure not to be wasting time on duplicate work.


\subsection{CI/CD Chain}
%A complete description of stages and tools included in the CI/CD chains.
%That is, including deployment and release of your systems.

%- Github Workflows
%- Digital Ocean

auto-release
-   schedulued release.yml
continous deployment
-  continous deployment.yml



\subsection{Repository Organization \& Development Strategy}
\subsubsection{Repository Organization}

We structurized our code in a mono-repository for two main reasons:
\begin{enumerate}
    \item It provides improved collaboration, as all developers are working on the same codebase.
    \item It simplifies version control as we only have a single history for all code.
\end{enumerate}
Another consideration was, that we didn't feel the need to divide it up, as the application is still relatively small. It would also add what we deemed to be unnecessary complexity by requiring different repositories to be cloud-hosted for cross-access, since the necessary references in our C\# code would no longer reside in the same .NET solution.


\subsubsection{Development Strategy}

We utilized feature branching as our strategy as it has multiple advantages that works very well with the DevOps way of working. An example of this is continous integration. The GitHub Actions make it simple to automate a require-to-pass test-suite before any feature branch is merged into the main one. Other advantages include:

\begin{itemize}
    \item It isolates changes to specific features, minimizing conflicts with separate work.
    \item It facilitates the code review process that in turn ensures quality ensurance.
    \item It provides a clear separation of work, which makes collaboration easier.
    \item It makes parallel development possible.
\end{itemize}

\noindent Our tasks were organized in GitHub Issues. This made it possible for each member to always be able to see which tasks were up for grabs, and who were working on what. It also acted as a task backlog giving an overview of the work that was yet to be done.

\noindent As the project progressed however, the tasks began including the whole group, and the Issues board became slightly irrelevant.

\subsection{Monitoring \& Logging}
How do you monitor your systems and what precisely do you monitor?
What do you log in your systems and how do you aggregate logs?

- Prometheus
    nuget package der lader prometheus snakked med vores program - laver metrics endpoint.
    enten er det ens application der sender metrics, ellers er det metrics-handleren der laver request og puller metrics løbende. vi beder løbende om metrics
    grafan visualizeer prometheus
    what precisely do we monitor: 
- Digital Ocean:
    -   hardware monitoring af CPU og memory
- Grafana


Hvad kunne vi have monitoret? Hvorfor har vi valgt det vi har valgt? Hvad kan man bruge det til? Har vi brugt det?
Fx: Active Users metric, kan vise hvordan systemet klarer sig, sammenhold med hvor mange brugere. Også ift. scaling og load-balancing.

we use - sink direkte fra .NET direkte ind til (serilog - logging framework). en sink, der hvor alle logs glider hen. Den sink er sat til 
sender det videre til elastisearch direkte
kun logs fra .Net så vi kan formattere det her og fortælle hvad vi gerne vil logge og hvordan det skal se ud direkte i serilog. 
der er formartering i serverens programfil. 
triggers i middleware. når vi modtager request bliver der lavet en log, når den er færdig med at processe - værktøjer der laver logs for os. 

alle exceptions
request til controllers
masse andre ting der er I middleware - 


\subsection{Security Assesment}
Brief results of the security assessment.



--------------------

Our security assessment consisted of a penetration test - Zed Attack Proxy, or ZAP, that showed some flaws in our program. Most seemed to be relatively simple to fix, and none seemed to be too severe in nature:
\begin{itemize}
    \item No Anti-CSRF tokens were found in a HTML submission form
    \item Passive (90022 - Application Error Disclosure)
    \item Content Security Policy (CSP) Header Not Set
    \item Missing Anti-clickjacking Header
    \item X-Content-Type-Options Header Missing
    \item Hidden File Found
    \item Vulnerable JS Library
    \item XSLT Injection might be possible.
    \item Cloud Metadata Potentially Exposed
\end{itemize}

\noindent These items resulted in GitHub Issues; some of which were dealt with while others were down-prioritized.

\subsection{Scaling \& Load Balancing}
%Applied strategy for scaling and load balancing. 



%loadbalancer: nginx, reverse proxy, 

%kinda docker-swarm 
%- nginx delay?
%- det var meningen at den skulle tage sig af redirection, men det virkede ikke som om at vi kunne det.
%- Det interne netværk skulle have tilladt redirection, men gjorde det ikke 

\subsection{AI Assistance}
%In case you have used AI assistants for writing code during your project or to write the report:
%Explain which system(s) you used during the project.
%Reflect on how it supported/hindered your process.
%In essence, it has to be clear how code or other artifacts come from idea into the running system and everything that happens on the way.



\noindent As developers we have embraced the tool that is AI. We have used ChatGPT as a sparring partner whenever we got a task we did not know how to solve, or when we got stuck. It has been mostly great at providing ideas or starting points, but it has probably been at its best when troubleshooting or debugging. If nothing else it has provided the services of a rubber duck.

\noindent We have also used it in the formulation of documents like the SLA agreement, as it is mostly boilerplate text anyways. We provided it with some information about our system and what guarantees we could give, and it returned a rough draft we could finalize.

\noindent Finally, one team member has been using GitHub Copilot as an advanced intellisense tool, but not for making several lines of code.